\documentclass[
	% -- opções da classe memoir --
	article,			% indica que é um artigo acadêmico
	11pt,				% tamanho da fonte
	oneside,			% para impressão apenas no recto. Oposto a twoside
	a4paper,			% tamanho do papel. 
	twocolumn,
	% -- opções da classe abntex2 --
	%chapter=TITLE,		% títulos de capítulos convertidos em letras maiúsculas
	%section=TITLE,		% títulos de seções convertidos em letras maiúsculas
	%subsection=TITLE,	% títulos de subseções convertidos em letras maiúsculas
	%subsubsection=TITLE % títulos de subsubseções convertidos em letras maiúsculas
	% -- opções do pacote babel --
	english,			% idioma adicional para hifenização
	brazil,				% o último idioma é o principal do documento
	sumario=tradicional
	]{abntex2}

\usepackage{cmap}	
\usepackage{lmodern}	
\usepackage[T1]{fontenc}	
\usepackage[utf8]{inputenc}		
\usepackage{lastpage}		
\usepackage{color}	
\usepackage{graphicx}	
\usepackage{units}
\usepackage[brazilian,hyperpageref]{backref}
\usepackage[num]{abntex2cite}
\usepackage{bold-extra}
\usepackage{eso-pic}
\usepackage{indentfirst, microtype}		
\usepackage{caption,subcaption,float, dcolumn}
\usepackage{amsfonts,amsmath,amssymb}
\usepackage{enumerate}
\usepackage{titlesec}

\usepackage{multirow}


\usepackage{algpseudocode}
\usepackage{algorithm}

\floatname{algorithm}{Algoritmo}
\renewcommand{\algorithmicrequire}{\textbf{Input:}}
\renewcommand{\algorithmicensure}{\textbf{Output:}}

\usepackage{svg}

\usepackage{listings}
\usepackage[newfloat]{minted}
% \usemintedstyle{friendly}
\citebrackets[]

\usemintedstyle{friendly}
\newenvironment{code}{\captionsetup{type=listing}}{}
\SetupFloatingEnvironment{listing}{name=Código}

\newcommand{\algorithmautorefname}{Algoritmo}
\renewcommand{\backrefpagesname}{Citado na(s) página(s):~}
\renewcommand{\backref}{}
\renewcommand*{\backrefalt}[4]{
	\ifcase #1 %
		Nenhuma citação no texto.%
	\or
		Citado na página #2.%
	\else
		Citado #1 vezes nas páginas #2.%
	\fi}%
% ---

\definecolor{blue}{rgb}{0, .226, .437}
\makeatletter
\hypersetup{
pdftitle={Circuito digital CMOS para controle do fator de qualidade de um filtro passa-banda ativo sintonizável}, 
		pdfauthor={Alef de Oliveira Santos},
    	pdfsubject={TCC2},
	    pdfcreator={LaTeX with abnTeX2},
        pdfkeywords={Fator Q. Controle digital. Métodos numéricos. RTL. Verilog.},
		colorlinks=true,       		% false: boxed links; true: colored links
    	linkcolor=blue,          	% color of internal links
    	citecolor=blue,        		% color of links to bibliography
    	filecolor=magenta,      		% color of file links
		urlcolor=blue,
		bookmarksdepth=4
}
\makeatother
\setlength{\parindent}{1.3cm}
\setlength{\parskip}{0.2cm}  
\makeindex

\renewcommand{\listalgorithmname}{Lista de algoritmos}

\titleformat{\chapter}     {\normalfont\normalsize\bfseries}{\thechapter}     				 {1em}{}
\titleformat{\section}     {\normalfont\normalsize}{\thechapter.\thesection}     			 {1em}{}
\titleformat{\subsection}  {\normalfont\normalsize}{\thechapter.\thesection.\thesubsection}  {1em}{}


\setlrmarginsandblock{1.5cm}{1.5cm}{*}
\setulmarginsandblock{1.5cm}{2.5cm}{*}
\checkandfixthelayout
