%=======PACOTES FUNDAMENTAIS================%

%9pt -> tamanho da fonte
%compress -> comprime o texto
% aspectratio -> razão largura x altura (16:9, 4:3, 1:1 ...)

\documentclass
    [9pt,
    compress,
    xcolor=svgnames,
    aspectratio=169]{beamer}

\usetheme
    [
    progressbar=frametitle,
    titleformatframe=smallcaps,
    numbering=counter
    ]{metropolis}
    
\useoutertheme[section=true, subsection=true]{miniframes} % inserir uma página indicando início de seção, subsection = true insere uma página pra subseção também

\usepackage{palatino}  % fonte principal - compilador PdfLaTeX
%\usepackage{lmodern} % fonte alternativa - compilador PdfLaTeX
\usepackage{courier} %fonte secundária
\usefonttheme[onlymath]{serif} % fonte matemática SERIFADA PELO AMOR DE DEUS

\usepackage[english, brazil]{babel} % idiomas
\usepackage{microtype}
\usepackage{listings} % códigos-fonte
\usepackage{ragged2e} % permite texto não-justificado (padrão)
\usepackage[utf8]{inputenc} %acentos e cedilha
\usepackage[T1]{fontenc} %acentos "copiáveis"
\usepackage{soul} %decoração de texto (underline, strike, ..)
% \usepackage[brazilian,hyperpageref]{backref}	 % Paginas com as citações na bibliografia
\usepackage[alf,abnt-full-initials=yes]{abntex2cite}	% Citações padrão ABNT

\usepackage{verbatim}



% \renewcommand{\backrefpagesname}{Citado na(s) página(s):~}
% \renewcommand{\backref}{}
% \renewcommand*{\backrefalt}[4]{
% 	\ifcase #1 %
% 		Nenhuma citação no texto.%
% 	\or
% 		Citado na página #2.%
% 	\else
% 		Citado #1 vezes nas páginas #2.%
% 	\fi}%



%====================tabelas================================%
\usepackage{etoolbox}
\usepackage{tcolorbox}
\usepackage{tabularx}
\usepackage{colortbl}
\usepackage{relsize}
\usepackage{multicol}
\usepackage{booktabs}

%===================gráficos=================================%
\usepackage{graphicx}
\usepackage{float}
\usepackage{caption}
\usepackage{subcaption}


%===================listas==========================%
\usepackage{enumerate}
\usepackage{array}


%==================matemática====================%
\usepackage{amsmath,amssymb}
\usepackage{pgfplots}
\usepgfplotslibrary{dateplot} % nunca nem vi, cuidado


%================espaçamentos====================%
\usepackage{geometry}
\usepackage{xspace}
\usepackage{setspace}
\usepackage{multicol}

%===================cores===========================%
\usepackage{xcolor}


%fg = foreground = plano principal
%bg = brackground = plano de fundo

\xdefinecolor{mainColor1}{HTML}{003A70}
\xdefinecolor{secondaryColor1}{HTML}{E5E5E5}
\xdefinecolor{secondaryColor2}{rgb}{0.6, 0.6, 0.6}

\setbeamercolor{progress bar}{fg=red}
\setbeamercolor{title separator}{fg=red}
\setbeamercolor{structure}{fg=black}
\setbeamercolor{normal text}{fg=black!87}
\setbeamercolor{alerted text}{fg=DarkRed!60!Gainsboro}
\setbeamercolor{example text}{fg=Maroon!70!Coral}
\setbeamercolor{palette primary}{bg=mainColor1, fg=secondaryColor1}
\setbeamercolor{palette secondary}{bg=mainColor1!50, fg=secondaryColor1}
\setbeamercolor{palette tertiary}{bg=NavyBlue!40!Black, fg= white}
\setbeamercolor{section in toc}{fg=NavyBlue!40!Black}


% Tem algumas cores padrão do LaTeX, desculpa a mistura de cor personalizada e padrão mas mo preguiça mudar

%====================blocos==============================%
\setbeamertemplate{blocks}[rectangle]

\colorlet{normalTitleBlockColor}{mainColor1}
\colorlet{normalBlockColor}{secondaryColor1}
\colorlet{alertTitleBlockColor}{red}
\colorlet{alertBlockColor}{secondaryColor1}
\colorlet{blockBodyTextColor}{NavyBlue!40!Black}

\setbeamercolor*{block title}{
  fg=white,
  bg=normalTitleBlockColor}
\setbeamercolor*{block body}{
  fg=blockBodyTextColor,
  bg=normalBlockColor}

\setbeamercolor*{block title alerted}{
  fg=white,
  bg=alertTitleBlockColor}
\setbeamercolor*{block body alerted}{
  fg=blockBodyTextColor,
  bg=alertBlockColor}
  
\AtBeginEnvironment{block}{
  \setbeamercolor{itemize item}{fg=normalTitleBlockColor!70}
  \setbeamercolor{enumerate item}{fg=normalTitleBlockColor!70}
} % troca as cores das bolinhas de enumerate e itemize de acordo com o tipo do bloco

\AtBeginEnvironment{alertblock}{
  \setbeamercolor{itemize item}{fg=alertTitleBlockColor!70}
  \setbeamercolor{enumerate item}{fg=alertTitleBlockColor!70}
} % troca as cores das bolinhas de enumerate e itemize de acordo com o tipo do bloco



%=-=-==============itemize,enumerate==================%

\setbeamercolor{itemize item}{fg=mainColor1!70}
\setbeamercolor{description item}{fg=mainColor1!70}

\usecolortheme[named=mainColor1]{structure} %enumerate

\useinnertheme{circles} % bolinhas ou {rectangles} nos enumerates e itemize

\usepackage{hyperref}
\hypersetup{
	bookmarks=true,     
	pdftoolbar=true,
	pdfmenubar=true,     
	pdffitwindow=true,
 	pdftitle={\@title},   
 	pdfauthor={\@author},   
	colorlinks=false,
%	usar as cores é meio arriscado, deixa à cargo do documento
% 	linkcolor=mainColor1,
% 	urlcolor=mainColor1,
% 	citecolor=blue,        
% 	filecolor=cyan,
}




% \setbeamertemplate{section in toc}[sections numbered] seções numeradas sem a bolinha dahora


%======================PERIGO-NÃO-MEXA==============================================================================================================================================================%

\setbeamertemplate{footline} 
  {%
    \begin{beamercolorbox}[colsep=1.5pt]{upper separation line foot}
    \end{beamercolorbox}
    \begin{beamercolorbox}
    [
    ht=2.5ex,
    dp=1.125ex,
    leftskip=.3cm,
    rightskip=.3cm plus1fil
    ]{title in head/foot}%
      \textcolor{white}{\insertshorttitle}%
      \hfill  \hspace{-2cm}\insertshortinstitute \hfill%
      {\usebeamerfont{frame number}\usebeamercolor[fg]{frame number}\insertframenumber~\frameofframes~\inserttotalframenumber}
    \end{beamercolorbox}%
    \begin{beamercolorbox}[colsep=1.5pt]{lower separation line foot}
    \end{beamercolorbox}
  }
\makeatother



%=====================================códigos===============================================================%
\usepackage{listings}

\definecolor{backcolour}{RGB}{245,245,245}
\definecolor{commentcolour}{RGB}{0,128,0}
\definecolor{keywordcolour}{RGB}{249,38,114}
\definecolor{stringcolour}{RGB}{255,102,0}

\renewcommand\lstlistingname{Código} 
\lstdefinestyle{padrao}{
    backgroundcolor=\color{backcolour},   
    commentstyle=\color{commentcolour},
    keywordstyle=\color{keywordcolour}\bfseries,
    numberstyle=\tiny\color{black},
    stringstyle=\color{stringcolour},
    emphstyle=\color{mainColor1},
    basicstyle=\footnotesize\ttfamily,
    breakatwhitespace=false,
    breaklines=true,                 
    captionpos=t,                    
    keepspaces=true,                 
    numbers=left,                    
    numbersep=5pt,                  
    showspaces=false,                
    showstringspaces=false,
    showtabs=false,                  
    tabsize=2
}
\lstset{literate=
  {á}{{\'a}}1 {é}{{\'e}}1 {í}{{\'i}}1 {ó}{{\'o}}1 {ú}{{\'u}}1
  {Á}{{\'A}}1 {É}{{\'E}}1 {Í}{{\'I}}1 {Ó}{{\'O}}1 {Ú}{{\'U}}1
  {à}{{\`a}}1 {è}{{\`e}}1 {ì}{{\`i}}1 {ò}{{\`o}}1 {ù}{{\`u}}1
  {À}{{\`A}}1 {È}{{\'E}}1 {Ì}{{\`I}}1 {Ò}{{\`O}}1 {Ù}{{\`U}}1
  {ä}{{\"a}}1 {ë}{{\"e}}1 {ï}{{\"i}}1 {ö}{{\"o}}1 {ü}{{\"u}}1
  {Ä}{{\"A}}1 {Ë}{{\"E}}1 {Ï}{{\"I}}1 {Ö}{{\"O}}1 {Ü}{{\"U}}1
  {â}{{\^a}}1 {ê}{{\^e}}1 {î}{{\^i}}1 {ô}{{\^o}}1 {û}{{\^u}}1
  {Â}{{\^A}}1 {Ê}{{\^E}}1 {Î}{{\^I}}1 {Ô}{{\^O}}1 {Û}{{\^U}}1
  {ã}{{\~a}}1 {ẽ}{{\~e}}1 {ĩ}{{\~i}}1 {õ}{{\~o}}1 {ũ}{{\~u}}1
  {Ã}{{\~A}}1 {Ẽ}{{\~E}}1 {Ĩ}{{\~I}}1 {Õ}{{\~O}}1 {Ũ}{{\~U}}1
  {œ}{{\oe}}1 {Œ}{{\OE}}1 {æ}{{\ae}}1 {Æ}{{\AE}}1 {ß}{{\ss}}1
  {ű}{{\H{u}}}1 {Ű}{{\H{U}}}1 {ő}{{\H{o}}}1 {Ő}{{\H{O}}}1
  {ç}{{\c c}}1 {Ç}{{\c C}}1 {ø}{{\o}}1 {å}{{\r a}}1 {Å}{{\r A}}1
  {€}{{\euro}}1 {£}{{\pounds}}1 {«}{{\guillemotleft}}1
  {»}{{\guillemotright}}1 {ñ}{{\~n}}1 {Ñ}{{\~N}}1 {¿}{{?`}}1 {¡}{{!`}}1 
}



%===========================comandos personalizados=============%

\captionsetup{position=top} 
\renewcommand\emph[1]{\textbf{\textcolor{mainColor1}{#1}}}
\newcommand{\frameofframes}{/}
\newcommand{\setframeofframes}[1]{\renewcommand{\frameofframes}{#1}}
\newcommand{\themename}{\textbf{\textsc{bluetemp}\xspace}}%metropolis}}\xspace}
\apptocmd{\frame}{}{\justifying}{}


\usepackage{listings}
\lstset{
  basicstyle=\ttfamily,
  mathescape,
  breaklines=true,
  breakatwhitespace=true,
}

\usepackage[newfloat]{minted}

\setminted{
    linenos=true,
    autogobble,
    breaklines,
    fontsize=\footnotesize
}

\usemintedstyle{friendly}
\newenvironment{code}
    {\captionsetup{type=listing}}{}
\SetupFloatingEnvironment{listing}{name=Código}

\usepackage{algpseudocode}
\usepackage[ruled]{algorithm}

\renewcommand{\algorithmicrequire}{\textbf{Input:}}
\renewcommand{\algorithmicensure}{\textbf{Output:}}

\floatname{Algoritmo}{algorithm}
\newcommand{\algorithmautorefname}{Algoritmo}
\citebrackets[]

\setkeys{Gin}{width=\textwidth}




\makeindex
% ---

% ---
% Altera as margens padrões
% ---
\titleformat{\chapter}     {\normalfont\normalsize\bfseries}{\thechapter}     				 {1em}{}
\titleformat{\section}     {\normalfont\normalsize}{\thechapter.\thesection}     			 {1em}{}
\titleformat{\subsection}  {\normalfont\normalsize}{\thechapter.\thesection.\thesubsection}  {1em}{}
% ---

% --- 
% Espaçamentos entre linhas e parágrafos 
% --- 

% O tamanho do parágrafo é dado por:
\setlength{\parindent}{1.3cm}

% Controle do espaçamento entre um parágrafo e outro:
\setlength{\parskip}{0.2cm}  % tente também \onelineskip

% Espaçamento simples
\SingleSpacing


% ----
% Início do documento
% ----
\begin{document}

% Seleciona o idioma do documento (conforme pacotes do babel)
%\selectlanguage{english}
\selectlanguage{brazil}

% Retira espaço extra obsoleto entre as frases.
\frenchspacing 

% ----------------------------------------------------------
% ELEMENTOS PRÉ-TEXTUAIS
% ----------------------------------------------------------

%---
%
% Se desejar escrever o artigo em duas colunas, descomente a linha abaixo
% e a linha com o texto ``FIM DE ARTIGO EM DUAS COLUNAS''.
% \twocolumn[    		% INICIO DE ARTIGO EM DUAS COLUNAS
%
%---

% página de titulo principal (obrigatório)
% \maketitle
% titulo em outro idioma (opcional)

\title{Circuito digital CMOS para controle do fator de qualidade de um filtro passa-banda ativo sintonizável}

\twocolumn[
	\begin{center}
		% % \begin{minipage}[H]{0.4}
		% % 	\begin{figure*}[]
		% % 		\centering
		% % 		\includegraphics[H]{./unifeilogo.eps}
				
		% % 	\end{figure*}
			
		% % \end{minipage}
		% \begin{minipage}
			
		% \end{minipage}
		\Large 
		 	Trabalho de conclusão de curso \\
		 	Universidade Federal de Itajubá - \textit{Campus} Theodomiro Carneiro Santiago\\
			Instituto de Ciências Tecnológicas - ICT\\
			\large
			Engenharia de Computação
		
		\normalsize
			
			
			\vspace*{1cm}
			\textbf{\thetitle}
			\vspace*{1cm}

			Alef de Oliveira Santos${}^*$, Dean Bicudo Karolak${}^*$, Paulo Márcio Moreira e Silva${}^*$

			\vspace*{0.5cm}

			${}^*$\textit{Universidade Federal de Itajubá - Campus Theodomiro Carneiro Santiago \\ Rua Rua Irmã Ivone Drumond, 200 - Distrito Industrial II - 35903-087 \\
			Itabira, Minas Gerais, Brasil} \\

			\vspace*{0.5cm}

			\textit{E-mail}: 
				\href{mailto:alef_santos@unifei.edu.br}{\verb|alef\_santos@unifei.edu.br}, 
				\href{mailto:dean.karolak@unifei.edu.br}{\verb|dean.karolak@unifei.edu.br},
				\href{mailto:paulo.silva@unifei.edu.br }{\verb|paulo.silva@unifei.edu.br } 
		
		
	\end{center}

	\noindent

	\textbf{Abstract:}\\

	\textbf{Keywords:}\\

	\textbf{Resumo:} 
	O controle do fator de qualidade ($Q$) em um filtro passa banda é uma maneira direta de controlar a seletividade deste filtro, e utilizando um circuito ativo é possível controlá-lo. Para controlar o $Q$ de um sistema ressonante com um circuito digital utiliza-se neste trabalho técnicas de computação numérica para aproximação de funções não-lineares, uma vez que os parâmetros do filtro \textit{on-chip} não são manipuláveis além da corrente que controla o $Q$. Assim, o objetivo deste trabalho é projetar um circuito digital sintetizável capaz de: receber como entrada o $Q$  desejado e medido de um filtro passa banda e controlar digitalmente esse valor de $Q$ do filtro. Além disso, três diferentes métodos numéricos de controle serão estudados e implementados para comparação e escolha do melhor método de convergência. \\
	\textbf{Palavras-Chave:}\\
]


% resumo em português
% \begin{resumoumacoluna}
%  Conforme a ABNT NBR 6022:2018, o resumo no idioma do documento é elemento obrigatório. 
%  Constituído de uma sequência de frases concisas e objetivas e não de uma 
%  simples enumeração de tópicos, não ultrapassando 250 palavras, seguido, logo 
%  abaixo, das palavras representativas do conteúdo do trabalho, isto é, 
%  palavras-chave e/ou descritores, conforme a NBR 6028. (\ldots) As 
%  palavras-chave devem figurar logo abaixo do resumo, antecedidas da expressão 
%  Palavras-chave:, separadas entre si por ponto e finalizadas também por ponto.
 
%  \vspace{\onelineskip}
 
%  \noindent
%  \textbf{Palavras-chave}: latex. abntex. editoração de texto.
% \end{resumoumacoluna}


% % resumo em inglês
% \renewcommand{\resumoname}{Abstract}
% \begin{resumoumacoluna}
%  \begin{otherlanguage*}{english}
%    According to ABNT NBR 6022:2018, an abstract in foreign language is optional.

%    \vspace{\onelineskip}
 
%    \noindent
%    \textbf{Keywords}: latex. abntex.
%  \end{otherlanguage*}  
% \end{resumoumacoluna}

% % ]  				% FIM DE ARTIGO EM DUAS COLUNAS
% % ---

% \begin{center}\smaller
% \textbf{Data de submissão e aprovação}: elemento obrigatório. Indicar dia, mês e ano

% \textbf{Identificação e disponibilidade}: elemento opcional. Pode ser indicado 
% o endereço eletrônico, DOI, suportes e outras informações relativas ao acesso.
% \end{center}

% ----------------------------------------------------------
% ELEMENTOS TEXTUAIS
% ----------------------------------------------------------
\textual

% ----------------------------------------------------------
% Introdução
% ----------------------------------------------------------
\chapter{Introdução}

% o que é o quê e por quê ele é importante?
% como medir o Q
% [ENFASE] como controlar o Q 
% 

Este trabalho trata de um sistema eletrônico que recebe e controla o fator de qualidade ($Q$) de um circuito eletrônico ressonante. O circuito proposto utiliza-se de técnicas de computação numérica para controlar o fator de qualidade através de uma corrente de referência injetada no sistema ressonante. Neste trabalho serão realizados circuitos digitais periféricos para a determinação do valor de $Q$ medido de um filtro passa banda ativo, bem como, um circuito digital para controle e aproximação do $Q$ desejado. Em relação ao controle e aproximação de Q, serão comparados os métodos numéricos da Bisseção, Secantes e Secantes com seleção de intervalo implementados. O sistema digital é projetado e implementado em Verilog tendo em mente a posterior fabricação em silício. Para efeitos de estudo e desenvolvimento, este projeto em ASIC utiliza uma tecnologia GPDK de 45nm.
% ----------------------------------------------------------
% ELEMENTOS PÓS-TEXTUAIS
% ----------------------------------------------------------
\postextual

% ----------------------------------------------------------
% Referências bibliográficas
% ----------------------------------------------------------
\bibliography{ref}

\end{document}
