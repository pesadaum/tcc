%=======PACOTES FUNDAMENTAIS================%

%9pt -> tamanho da fonte
%compress -> comprime o texto
% aspectratio -> razão largura x altura (16:9, 4:3, 1:1 ...)

\documentclass
    [9pt,
    compress,
    xcolor=svgnames,
    aspectratio=169]{beamer}

\usetheme
    [
    progressbar=frametitle,
    titleformatframe=smallcaps,
    numbering=counter
    ]{metropolis}
    
\useoutertheme[section=true, subsection=true]{miniframes} % inserir uma página indicando início de seção, subsection = true insere uma página pra subseção também

\usepackage{palatino}  % fonte principal - compilador PdfLaTeX
%\usepackage{lmodern} % fonte alternativa - compilador PdfLaTeX
\usepackage{courier} %fonte secundária
\usefonttheme[onlymath]{serif} % fonte matemática SERIFADA PELO AMOR DE DEUS

\usepackage[english, brazil]{babel} % idiomas
\usepackage{microtype}
\usepackage{listings} % códigos-fonte
\usepackage{ragged2e} % permite texto não-justificado (padrão)
\usepackage[utf8]{inputenc} %acentos e cedilha
\usepackage[T1]{fontenc} %acentos "copiáveis"
\usepackage{soul} %decoração de texto (underline, strike, ..)
% \usepackage[brazilian,hyperpageref]{backref}	 % Paginas com as citações na bibliografia
\usepackage[alf,abnt-full-initials=yes]{abntex2cite}	% Citações padrão ABNT

\usepackage{verbatim}



% \renewcommand{\backrefpagesname}{Citado na(s) página(s):~}
% \renewcommand{\backref}{}
% \renewcommand*{\backrefalt}[4]{
% 	\ifcase #1 %
% 		Nenhuma citação no texto.%
% 	\or
% 		Citado na página #2.%
% 	\else
% 		Citado #1 vezes nas páginas #2.%
% 	\fi}%



%====================tabelas================================%
\usepackage{etoolbox}
\usepackage{tcolorbox}
\usepackage{tabularx}
\usepackage{colortbl}
\usepackage{relsize}
\usepackage{multicol}
\usepackage{booktabs}

%===================gráficos=================================%
\usepackage{graphicx}
\usepackage{float}
\usepackage{caption}
\usepackage{subcaption}


%===================listas==========================%
\usepackage{enumerate}
\usepackage{array}


%==================matemática====================%
\usepackage{amsmath,amssymb}
\usepackage{pgfplots}
\usepgfplotslibrary{dateplot} % nunca nem vi, cuidado


%================espaçamentos====================%
\usepackage{geometry}
\usepackage{xspace}
\usepackage{setspace}
\usepackage{multicol}

%===================cores===========================%
\usepackage{xcolor}


%fg = foreground = plano principal
%bg = brackground = plano de fundo

\xdefinecolor{mainColor1}{HTML}{003A70}
\xdefinecolor{secondaryColor1}{HTML}{E5E5E5}
\xdefinecolor{secondaryColor2}{rgb}{0.6, 0.6, 0.6}

\setbeamercolor{progress bar}{fg=red}
\setbeamercolor{title separator}{fg=red}
\setbeamercolor{structure}{fg=black}
\setbeamercolor{normal text}{fg=black!87}
\setbeamercolor{alerted text}{fg=DarkRed!60!Gainsboro}
\setbeamercolor{example text}{fg=Maroon!70!Coral}
\setbeamercolor{palette primary}{bg=mainColor1, fg=secondaryColor1}
\setbeamercolor{palette secondary}{bg=mainColor1!50, fg=secondaryColor1}
\setbeamercolor{palette tertiary}{bg=NavyBlue!40!Black, fg= white}
\setbeamercolor{section in toc}{fg=NavyBlue!40!Black}


% Tem algumas cores padrão do LaTeX, desculpa a mistura de cor personalizada e padrão mas mo preguiça mudar

%====================blocos==============================%
\setbeamertemplate{blocks}[rectangle]

\colorlet{normalTitleBlockColor}{mainColor1}
\colorlet{normalBlockColor}{secondaryColor1}
\colorlet{alertTitleBlockColor}{red}
\colorlet{alertBlockColor}{secondaryColor1}
\colorlet{blockBodyTextColor}{NavyBlue!40!Black}

\setbeamercolor*{block title}{
  fg=white,
  bg=normalTitleBlockColor}
\setbeamercolor*{block body}{
  fg=blockBodyTextColor,
  bg=normalBlockColor}

\setbeamercolor*{block title alerted}{
  fg=white,
  bg=alertTitleBlockColor}
\setbeamercolor*{block body alerted}{
  fg=blockBodyTextColor,
  bg=alertBlockColor}
  
\AtBeginEnvironment{block}{
  \setbeamercolor{itemize item}{fg=normalTitleBlockColor!70}
  \setbeamercolor{enumerate item}{fg=normalTitleBlockColor!70}
} % troca as cores das bolinhas de enumerate e itemize de acordo com o tipo do bloco

\AtBeginEnvironment{alertblock}{
  \setbeamercolor{itemize item}{fg=alertTitleBlockColor!70}
  \setbeamercolor{enumerate item}{fg=alertTitleBlockColor!70}
} % troca as cores das bolinhas de enumerate e itemize de acordo com o tipo do bloco



%=-=-==============itemize,enumerate==================%

\setbeamercolor{itemize item}{fg=mainColor1!70}
\setbeamercolor{description item}{fg=mainColor1!70}

\usecolortheme[named=mainColor1]{structure} %enumerate

\useinnertheme{circles} % bolinhas ou {rectangles} nos enumerates e itemize

\usepackage{hyperref}
\hypersetup{
	bookmarks=true,     
	pdftoolbar=true,
	pdfmenubar=true,     
	pdffitwindow=true,
 	pdftitle={\@title},   
 	pdfauthor={\@author},   
	colorlinks=false,
%	usar as cores é meio arriscado, deixa à cargo do documento
% 	linkcolor=mainColor1,
% 	urlcolor=mainColor1,
% 	citecolor=blue,        
% 	filecolor=cyan,
}




% \setbeamertemplate{section in toc}[sections numbered] seções numeradas sem a bolinha dahora


%======================PERIGO-NÃO-MEXA==============================================================================================================================================================%

\setbeamertemplate{footline} 
  {%
    \begin{beamercolorbox}[colsep=1.5pt]{upper separation line foot}
    \end{beamercolorbox}
    \begin{beamercolorbox}
    [
    ht=2.5ex,
    dp=1.125ex,
    leftskip=.3cm,
    rightskip=.3cm plus1fil
    ]{title in head/foot}%
      \textcolor{white}{\insertshorttitle}%
      \hfill  \hspace{-2cm}\insertshortinstitute \hfill%
      {\usebeamerfont{frame number}\usebeamercolor[fg]{frame number}\insertframenumber~\frameofframes~\inserttotalframenumber}
    \end{beamercolorbox}%
    \begin{beamercolorbox}[colsep=1.5pt]{lower separation line foot}
    \end{beamercolorbox}
  }
\makeatother



%=====================================códigos===============================================================%
\usepackage{listings}

\definecolor{backcolour}{RGB}{245,245,245}
\definecolor{commentcolour}{RGB}{0,128,0}
\definecolor{keywordcolour}{RGB}{249,38,114}
\definecolor{stringcolour}{RGB}{255,102,0}

\renewcommand\lstlistingname{Código} 
\lstdefinestyle{padrao}{
    backgroundcolor=\color{backcolour},   
    commentstyle=\color{commentcolour},
    keywordstyle=\color{keywordcolour}\bfseries,
    numberstyle=\tiny\color{black},
    stringstyle=\color{stringcolour},
    emphstyle=\color{mainColor1},
    basicstyle=\footnotesize\ttfamily,
    breakatwhitespace=false,
    breaklines=true,                 
    captionpos=t,                    
    keepspaces=true,                 
    numbers=left,                    
    numbersep=5pt,                  
    showspaces=false,                
    showstringspaces=false,
    showtabs=false,                  
    tabsize=2
}
\lstset{literate=
  {á}{{\'a}}1 {é}{{\'e}}1 {í}{{\'i}}1 {ó}{{\'o}}1 {ú}{{\'u}}1
  {Á}{{\'A}}1 {É}{{\'E}}1 {Í}{{\'I}}1 {Ó}{{\'O}}1 {Ú}{{\'U}}1
  {à}{{\`a}}1 {è}{{\`e}}1 {ì}{{\`i}}1 {ò}{{\`o}}1 {ù}{{\`u}}1
  {À}{{\`A}}1 {È}{{\'E}}1 {Ì}{{\`I}}1 {Ò}{{\`O}}1 {Ù}{{\`U}}1
  {ä}{{\"a}}1 {ë}{{\"e}}1 {ï}{{\"i}}1 {ö}{{\"o}}1 {ü}{{\"u}}1
  {Ä}{{\"A}}1 {Ë}{{\"E}}1 {Ï}{{\"I}}1 {Ö}{{\"O}}1 {Ü}{{\"U}}1
  {â}{{\^a}}1 {ê}{{\^e}}1 {î}{{\^i}}1 {ô}{{\^o}}1 {û}{{\^u}}1
  {Â}{{\^A}}1 {Ê}{{\^E}}1 {Î}{{\^I}}1 {Ô}{{\^O}}1 {Û}{{\^U}}1
  {ã}{{\~a}}1 {ẽ}{{\~e}}1 {ĩ}{{\~i}}1 {õ}{{\~o}}1 {ũ}{{\~u}}1
  {Ã}{{\~A}}1 {Ẽ}{{\~E}}1 {Ĩ}{{\~I}}1 {Õ}{{\~O}}1 {Ũ}{{\~U}}1
  {œ}{{\oe}}1 {Œ}{{\OE}}1 {æ}{{\ae}}1 {Æ}{{\AE}}1 {ß}{{\ss}}1
  {ű}{{\H{u}}}1 {Ű}{{\H{U}}}1 {ő}{{\H{o}}}1 {Ő}{{\H{O}}}1
  {ç}{{\c c}}1 {Ç}{{\c C}}1 {ø}{{\o}}1 {å}{{\r a}}1 {Å}{{\r A}}1
  {€}{{\euro}}1 {£}{{\pounds}}1 {«}{{\guillemotleft}}1
  {»}{{\guillemotright}}1 {ñ}{{\~n}}1 {Ñ}{{\~N}}1 {¿}{{?`}}1 {¡}{{!`}}1 
}



%===========================comandos personalizados=============%

\captionsetup{position=top} 
\renewcommand\emph[1]{\textbf{\textcolor{mainColor1}{#1}}}
\newcommand{\frameofframes}{/}
\newcommand{\setframeofframes}[1]{\renewcommand{\frameofframes}{#1}}
\newcommand{\themename}{\textbf{\textsc{bluetemp}\xspace}}%metropolis}}\xspace}
\apptocmd{\frame}{}{\justifying}{}


\usepackage{listings}
\lstset{
  basicstyle=\ttfamily,
  mathescape,
  breaklines=true,
  breakatwhitespace=true,
}

\usepackage[newfloat]{minted}

\setminted{
    linenos=true,
    autogobble,
    breaklines,
    fontsize=\footnotesize
}

\usemintedstyle{friendly}
\newenvironment{code}
    {\captionsetup{type=listing}}{}
\SetupFloatingEnvironment{listing}{name=Código}

\usepackage{algpseudocode}
\usepackage[ruled]{algorithm}

\renewcommand{\algorithmicrequire}{\textbf{Input:}}
\renewcommand{\algorithmicensure}{\textbf{Output:}}

\floatname{Algoritmo}{algorithm}
\newcommand{\algorithmautorefname}{Algoritmo}
\citebrackets[]

\setkeys{Gin}{width=\textwidth}
